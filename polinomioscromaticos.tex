\documentclass[a4paper]{article}
\usepackage{graphicx}
\usepackage{fancyhdr}
\usepackage{tocloft}
\usepackage{lipsum}
\usepackage{amsmath}
\usepackage{amssymb}
\usepackage{float}
\usepackage{tcolorbox}

\title{Polinomios Cromáticos}
\author{Gaspar Giménez - Docente: Regina Muzzulini}
\date{\today}

\renewcommand{\cftsecleader}{\cftdotfill{\cftdotsep}}
\renewcommand{\cftsecfont}{\bfseries}
\renewcommand{\cftsecdotsep}{\cftdotsep}

\pagestyle{fancy}
\fancyhf{}
\renewcommand{\headrulewidth}{0pt}
\fancyfoot[C]{Instituto Politécnico Superior --- \thepage}

\newcommand{\highlight}[1]{\begin{tcolorbox}[colback=gray!10,colframe=gray!30]#1\end{tcolorbox}}

\begin{document}

\begin{titlepage}
    \centering
    \includegraphics[width=0.4\textwidth]{./logogrande.png}\par\vspace{1cm}
    {\scshape\LARGE Instituto Politécnico Superior \par}
    \vspace{1cm}
    {\huge\bfseries Polinomios Cromáticos\par}
    \vspace{2cm}
    {\Large\itshape Gaspar Giménez - Docente: Regina Muzzulini\par}
    \vfill
    {\large \today\par}
\end{titlepage}

\maketitle

\tableofcontents

\pagestyle{fancy}

\section{Introducción}

\subsection{De cuántas formas distintas puedo colorear un único grafo?}

\includegraphics[width=\textwidth]{./imagen1.jpg}

\textbf{Definición:} Llamaremos Número Cromático al menor número de colores necesarios para colorear un grafo $G$, denotado como $X(G)$.


\highlight{
¿Cómo se obtiene este número cromático?
Hay distintos casos:

\begin{enumerate}
    \item $X(K_n) = n$
    \item $X(L_n) = 2$ para todo $n \geq 2$ (Siendo $L$ un grafo lineal o camino)
    \item $X(N_n) = 1$ (Grafo vacío, es decir $n$ vértices y $0$ aristas)
    \item $X(C_n) = 2$ si $n$ es par, $3$ si $n$ es impar (Grafo circular o circuito)
    \item $X(R_n) = 4$ si $n$ es par, $3$ si $n$ es impar (Grafo rueda)
    \item $X(G) = 2$ si $G$ es bipartito
    \item $X(K_{r,s}) = 2$ con $K$ bipartito completo
\end{enumerate}
}

En casos donde el grafo no sea uno de estos, deberemos recurrir a algoritmos o distintas técnicas para encontrarlo.

\subsection{Polinomio Cromático}

\textbf{Definición:} Dado un grafo $G$ no dirigido y un $k$ natural $\geq 1$, llamamos polinomio cromático de $G$ a la función de $k$ que nos da el número de formas de colorear $G$ con $k$ colores.
\[ PG(k) = \text{polinomio cromático de } G \]

\highlight{\textbf{Propiedad 1)} Si $k < X(G)$ entonces no podré colorear el grafo, es decir $PG(k) = 0$.}

\highlight{\textbf{Propiedad 2)} Si $k \geq X(G)$ entonces podré colorearlo de al menos una forma, es decir $PG(k) \geq 1$.}

\highlight{\textbf{Propiedad 3)} Además, si $k < k'$ entonces $PG(k) < PG(k')$.}

Luego, por estas tres propiedades:

\highlight{\textbf{Conclusión 1)} $X(G)$ es el menor número natural para el cual $PG(k)$ no es nulo.}

\highlight{\textbf{Conclusión 2)} Si $G$ es un grafo no dirigido con componentes conexas $C_1, C_2, \ldots, C_r$ con $r \geq 1$, entonces 
\[ PG(k) = PG(C_1) \cdot PG(C_2) \cdot \ldots \cdot PG(C_r) \]}

\textbf{Algunos polinomios cromáticos conocidos:}
\begin{itemize}
    \item $K_3: t \cdot (t-1) \cdot (t-2)$
    \item $K_n: t \cdot (t-1) \cdot (t-2) \cdot \ldots \cdot (t-(n-1))$
    \item Árbol con $n$ vértices: $t \cdot (t-1)^{n-1}$
    \item Ciclo $C_n$: $(t-1)^n + (-1)^n \cdot (t-1)$
    \item Lineal $L_n$: $t^n$
\end{itemize}

\section{Unión de Grafos}

Llamamos unión de los grafos $G$ y $H$ al grafo $G \cup H = (V, E)$ donde $V = V(G) \cup V(H)$ y $E = E(G) \cup E(H)$.

\includegraphics[width=\textwidth]{./imagen4.jpg}

De esta forma podemos descomponer grafos que tengan un vértice en común, por ejemplo:

\includegraphics[width=\textwidth]{./imagen5.jpg}

¿Para qué nos sirve? Bueno, gracias a la descomposición de grafos podemos obtener:

\highlight{\textbf{Si $G$ y $H$ tienen un único vértice en común, se verifica que:}}
\[ \textbf{PG}_{G \cup H}(k) = \frac{\textbf{PG}_G(k) \cdot \textbf{PG}_H(k)}{k} \]

\highlight{\textbf{Si $G$ y $H$ tienen una única arista en común $e = (u,v)$, entonces se verifica que:}}
\[ \textbf{PG}_{G \cup H}(k) = \frac{\textbf{PG}_G(k) \cdot \textbf{PG}_H(k)}{k \cdot (k-1)} \]

\section{Diferencias de Grafos}

\textbf{Definición:} El grafo diferencia $G - a$ es un grafo que tiene los mismos vértices que $G$ y las mismas aristas que $G$ excepto la arista $a$ (eliminamos la arista, pero dejamos sus extremos).

\textbf{Definición:} El grafo cociente $G_a$ es un grafo en el cual identificamos los dos extremos de $a$ en uno solo. Por tanto, $G_a$ tiene un vértice menos que $G$ y la arista $a$ se elimina. Dos aristas no incidentes que tienen como extremos los vértices de $a$ en $G$, pasan a tener un vértice común en $G_a$, y dos aristas cualesquiera que formen un triángulo con $a$ en $G$ pasan a ser la misma arista en $G_a$.

\includegraphics[width=\textwidth]{./Grafo1.jpg}

\highlight{\textbf{Teorema:} $PG(k) = \textbf{PG}_{-a}(k) - \textbf{PG}_{a}(k)$}

\highlight{\textbf{Ejemplo:}} Si quisiéramos calcular, por ejemplo, el Polinomio cromático del grafo $G$ de la figura anterior:

\[ PG (k) = \textbf{PG}_{-a}(k) - \textbf{PG}_{a}(k) = \textbf{PL}_3 (k) - \textbf{PK}_3 (k) = k^3 - k \cdot (k-1) \cdot (k-2) \]

\end{document}
